\chapter{引言}
\label{chap:introduction}

\begin{figure}[htb]
    \centering
    \includegraphics[width=0.4\textwidth]{amazon.PNG}
    \caption[超级皇后]{%
      超级皇后走法,可以像皇后与骑士一样移动%
      \cite{wikiAmazon}。}
    \label{fig:superQueen}
  \end{figure}

\section{游戏背景与规则}
国际象棋影响深远受众广泛,在经历了漫长的传播与演变后形成了今日的规则。
A History of Chess 详细记载了其演变过程与各种游戏规则的变种与分支\cite{murray2015history}。在中世纪之后的很长一段时间内,皇后棋子(或者被称为Amazon)可以像现代皇后或骑士一样移动,只不过由于太过强力,相对平衡的现代皇后在国际象棋规则中作为正统胜出。
指导老师与笔者受其启发,构思出一款国际象棋衍生游戏—超级皇后对战。
规则描述如下:指定大小的棋盘(8x8,12x12等)有黑白两个超级皇后;超级皇后的落子规则可指定(皇后走法,骑士走法,混合走法),不可越位,不可攻击对方;每当超级皇后离开当前位置,该位置变为不可落子的死地;胜利条件为封死对方超级皇后的行动。相比于围棋和国际象棋,此游戏的状态与所涉及到的状态空间并不复杂

\begin{figure}[htb]
    \centering
    \includegraphics[width=1\textwidth]{rules.PNG}
    \caption[游戏规则]{%
      超级皇后对战游戏规则%
      }
    \label{fig:superQueenRules}
  \end{figure}


\section{本文工作}
我们构思出了一款国际象棋衍生游戏,并实现了对战框架,可进行人机对战与电脑AI对战。针对电脑AI玩家,我们设计了以下几种AI:(1)随机玩家(Random Player),顾名思义,其每步动作都基于当前可行走法进行随机选择,此玩家也将作为baseline测试其余AI是否有效;
(2)贪婪玩家(Greedy Player),依据评分函数,在每一步选择中都采取在当前状态下最好或最优(即最有利)的走法;(3)Alpha-beta剪枝玩家,采取带alpha-beta剪枝的极大极小化算法,在有限深度内博弈得到最佳走法\cite{russell2010artificial};(4)AlphaZero框架玩家,依据谷歌公司提出来的理论,即结合蒙特卡洛树搜索与深度神经网络进行强化学习的AI训练方法\cite{Silver1140}。
同时,针对AlphaZero框架,由于超级皇后对战与围棋、五子棋、国际象棋在棋面局势评估上具有显著不同,我们还针对其网络结构与初始化设置不同的有效性进行了探索研究。

\section{论文结构}
本文总共包含五章。在第一章中,我们介绍所构思的新游戏背景与规则。此外并介绍了本文的主要工作,最后介绍本文结构。

在第二章中,会介绍本文所设计的新游戏AI涉及到的理论,包括极大极小化算法,alpha-beta剪枝以及组成AlphaZero结构的蒙特卡洛树搜索,深度神经网络与强化学习等基本理论。

在第三章中,会介绍我们的对战框架设计与实际训练过程。

在第四章中,会介绍我们的训练结果与AI对战强度测试。在最后的第五章中,我们会对整个工作进行总结。

% \cite{Book:GV1996}
% \cite{wikiAmazon}