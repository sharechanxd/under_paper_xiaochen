\chapter*{致谢}
\addcontentsline{toc}{chapter}{致谢}
本科五年,一晃而去,日月笼中鸟,乾坤水上萍,别离之日,亦有不舍。前两年学会了思考,后三年学会了实践与取舍。

首先感谢悉心指导本论文的邵老师,虽然相处仅有半年,但是邵老师对学术问题的热情以及严谨治学的态度常常令处于毕业季而态度轻浮的我羞愧不已。邵老师对毕业论文的指导帮助我进一步树立了良好的写作习惯,想必在接下来的求学之路定能受益匪浅。也感谢邵老师给予的机会能让我在毕业前亲手设计一款游戏并写出一个具有开源意义的人工智能。即便未来不再从事这一方向,也将是一份扎实而快乐的经历回忆。
同时也感谢戴伟辉老师,他是我学术研究的启蒙老师,在戴老师的指导下我完成了第一份学术论文与第一份完整的科研项目,激起了我对科研的兴趣,直接促成了我从管理学院转去了大数据学院,也因此让我在两个学院获得了丰富而迥异的学习与工作经验,对世界与社会的认知能力大大提高。
还要感谢庄吓海老师,带我入了生物计算与医学影像的门,从此开始了对计算机视觉的研究,并因此获得了在腾讯与eBay实习的机会,也认识了更多超级棒的同事与老板。学界与业界的碰撞让我对人工智能产业有了更多的思考,也让我确定了下一阶段的目标。也很感谢组里合作过的学长学姐;也很感谢其他在复旦的老师,从他们身上我学到了方方面面。
除此之外,我还想感谢陈剑博士。在疫情期间,陈剑博士带领我做了不少具有相当影响力工作,作为后辈受到了陈剑博士的很多关怀与提携,也从陈博士身上学到了不少人生经验。高人指点,贵人相助,再次感谢各位老师、老板、学长学姐的教诲与指导!

感谢我的家人们从小到大一直以来的支持、无私的付出,大学五年尝试了非常多想做的事情,你们始终在各方面支持着我,尤其是妈妈对我很多任性要求的满足以及充分的信任,让我感受到被爱的美好;感谢小孙在我最难过的时候的陪伴,相处的日子总是如沐春风,未来的日子还有很多很多,请多指教。

还想感谢一下众多的狐朋狗友与亲亲姐妹们。在管院,在大数据,在学校内,在校外都认识了非常多有趣而深刻的人,各位追求自己人生中的美好的模样总是动人的,在学生时代如此,迈入社会后想必我们也会继续互相鞭策而共同进步。
在申请季与未来规划上对我产生了极大启发的王嘉顺,没有他的帮助想必我在申请季中会过得更加痛苦,而与他进行思维的碰撞总是火花四溅而愉悦的。还有我在大数据的三位室友,四个转降人共同度过了不少课业与情感难关。
还有许许多多的朋友,合意友来情不厌,知心人至话投机。感谢这一路帮助过我的人!
