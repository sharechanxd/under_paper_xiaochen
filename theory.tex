\chapter{理论分析}
\label{chap:theory}
在这一章,我们将会对涉及到的理论与算法原理进行介绍。
\section{博弈树}
具有竞争或对抗性质的行为称为博弈行为。在这类行为中,参加斗争或竞争的各方各自具有不同的目标或利益。为了达到各自的目标和利益,各方必须考虑对手的各种可能的行动方案,并力图选取对自己最为有利或最为合理的方案。比如日常生活中的下棋,打牌等。博弈论就是研究博弈行为中斗争各方是否存在着最合理的行为方案,以及如何找到这个合理的行为方案的数学理论和方法\cite{gt}。
而博弈树是博弈理论中表达一个博弈中各种后续可能性的树。完整博弈树(Complete Game Tree)从代表某个博弈情景的起始节点出发,向下延展出若干层的子节点直到博弈结束。下一层的子节点是基于其父节点博弈行为所导致的可能性。博弈树中形成的叶节点代表各种游戏结束的可能情形,例如井字游戏(Tic-Tac-Toe)会有26,830个叶节点\cite{NAU1982257,allis1994searching}。


博弈树在人工智能应用领域占有重要地位,在博弈游戏中选择最佳动作的一种方法便是使用某种树搜索算法,结合类似于极小化极大算法的规则来修剪树,从而搜索整个博弈树。例如在井字游戏中计算机可以很快速地找到最佳解并做出决策,但是对于象棋、围棋这一类状态空间复杂的大型博弈游戏,受限于计算机性能遍历完整博弈树不太现实,因此对这类游戏通常会采用部分博弈树(partial game tree)来进行搜索。典型的部分博弈树通常是限制博弈树的层数,并剔除不佳的步法(例如自杀),一般而言搜索的层数越多,能走出较佳步法的机会也越高\cite{coin12162}。

\subsection{极小化极大算法}
极小化极大算法是人工智能领域常见的搜索算法
\subsection{Alpha-beta 剪枝}

\subsection{蒙特卡洛树搜索}

\section{卷积神经网络模型}

\subsection{深度残差网络}

\section{强化学习}

\subsection{基于策略的算法}

\subsection{基于值的算法}